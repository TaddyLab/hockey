%!TEX root = book.tex

\chapter{Hockey Player Ability via Logistic Regression}

\noindent
{\bf Matt Taddy, Sen Tian, Robert B.~Gramacy}

\bigskip
Plus-minus is the traditional metric for wholistically evaluating player
contributions in hockey, but it measures a marginal effect, fails to account
for sample size, and is otherwise a noisy estimate.  A better approach would
control for who players are playing with, and against, and thereby produce a
partial player effect.  One way to do this, using the same resolution of data
as for plus-minus (a record of which players were on the ice for each goal) is
by modeling the log odds that a goal was scored for or against their team
while they (and other players) were on the ice.  The simplest example of such
a model is a logistic regression with a design matrix composed of player
indicators, and responses indicating which team scored the goal.  The
resulting partial player effects can then be backed out of the regression
coefficients estimated from the goals data. 

The problem with this setup is that it is ``too big'' for ordinary logistic
regression software, and the design matrices are too highly imbalanced to
obtain meaningful (low variance) estimates of player effects.  Over the span
of several seasons there may be thousands of players involved in tens of
thousands of goals, but the number of unique player configurations is small by
comparison. That's because players play with and against only a small fraction
of other players.  In most of the cases we've tried ordinary inference
algorithms, e.g., Fisher scoring, fail to converge.  Classical remedies, like
foward/backward stepwise selection via information criteria, involve too much
computation to be practical.

The situation is much improved, however, when set in the modern context of
pelalized logistic regression.  The sofware packages accompanying this new
methodolog---which have recently undergone several overhauls to exploit
parallel computing environemnts, sparse matrices, and other features of modern
data structures---have demonstrated an impressive ability to gracefully handle
large and highly imbalanced data sets.  The result is that it is now relatively
easy to entertain the calculation of partial player effects in hockey through
a relatively straightforward logistic regression formulation.  In this chapter
we will cover the basic setup, using goals data spanning more than a decade,
and discuss the interpretation of the effects on the same scale of the
original plus-minus.  

As a testament to the flexibility of the framework we will then embellish the
model to include predictors/data beyond goals.  We encorporate salary
information, team and coach effects, special teams and playoff indicators,
season effects, and interactions between all of the above.  We also explore a
similar analysis based on shots rather than goals, however we find far less
signal in this data despite the much larger sample size.  This is a somewhat
suprising results considering recent trends in hockey analytics.  We conclude
with thoughts on potential improvement beyond linear logistic modeling,
including modern machine learning classifiers such as those based on trees
(random forests, Bayesian additive regression trees, etc.) and kernel methods
(support vector machines, Gaussian processes, etc.)

\section{Introduction}

\section{The base model and implementation}

Discuss \cite{gramacy:jensen:taddy:2013}, the basic framework and implementation.

\section{A subsequently more mature model}

\section{Embellishments}

\section{An example}

\section{Conclusions}

\bibliographystyle{plain}
\bibliography{hockey}